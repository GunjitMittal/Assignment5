\let\negmedspace\undefined{}
\let\negthickspace\undefined{}
\documentclass[journal,12pt,twocolumn]{IEEEtran}
 \usepackage{gensymb}
 \usepackage{polynom}
\usepackage{amssymb}
\usepackage[cmex10]{amsmath}
\usepackage{amsthm}
 \usepackage{stfloats}
\usepackage{bm} 
 \usepackage{longtable}
 \usepackage{enumitem}
 \usepackage{mathtools}
 \usepackage{tikz}
 \usepackage[breaklinks=true]{hyperref}
\usepackage{listings}
\usepackage{color}                                            
\usepackage{array}                                            
\usepackage{longtable}                                        
\usepackage{calc}                                             
    \usepackage{multirow}                                         
    \usepackage{hhline}                                           
    \usepackage{ifthen}                                           
    \usepackage{lscape}     
\DeclareMathOperator*{\Res}{Res}
\DeclareMathOperator*{\equals}{=}
\renewcommand\thesection{\arabic{section}}
\renewcommand\thesubsection{\thesection.\arabic{subsection}}
\renewcommand\thesubsubsection{\thesubsection.\arabic{subsubsection}}
\renewcommand\thesectiondis{\arabic{section}}
\renewcommand\thesubsectiondis{\thesectiondis.\arabic{subsection}}
\renewcommand\thesubsubsectiondis{\thesubsectiondis.\arabic{subsubsection}}
\hyphenation{op-tical net-works semi-conduc-tor}
\def\inputGnumericTable{}                                 %%
\lstset{ 
frame=single,
breaklines=true,
columns=fullflexible
}
\begin{document}
\newtheorem{theorem}{Theorem}[section]
\newtheorem{problem}{Problem}
\newtheorem{proposition}{Proposition}[section]
\newtheorem{lemma}{Lemma}[section]
\newtheorem{corollary}[theorem]{Corollary}
\newtheorem{example}{Example}[section]
\newtheorem{definition}[problem]{Definition}
\newcommand{\BEQA}{\begin{eqnarray}}
\newcommand{\EEQA}{\end{eqnarray}}
\newcommand{\define}{\stackrel{\triangle}{=}}
\newcommand*\circled[1]{\tikz[baseline= (char.base)]{
    \node[shape=circle,draw,inner sep=2pt] (char) {#1};}}
\bibliographystyle{IEEEtran}
\providecommand{\mbf}{\mathbf}
\providecommand{\pr}[1]{\ensuremath{\Pr\left(#1\right)}}
\providecommand{\qfunc}[1]{\ensuremath{Q\left(#1\right)}}
\providecommand{\sbrak}[1]{\ensuremath{{}\left[#1\right]}}
\providecommand{\lsbrak}[1]{\ensuremath{{}\left[#1\right.]}}
\providecommand{\rsbrak}[1]{\ensuremath{{}\left[#1\right.]}}
\providecommand{\brak}[1]{\ensuremath{\left(#1\right)}}
\providecommand{\lbrak}[1]{\ensuremath{\left(#1\right.)}
\providecommand{\rbrak}[1]{\ensuremath{\left[#1\right.]}}}
\providecommand{\cbrak}[1]{\ensuremath{\left\{#1\right\}}}
\providecommand{\lcbrak}[1]{\ensuremath{\left\{#1\right.}}
\providecommand{\rcbrak}[1]{\ensuremath{\left.#1\right\}}}
\theoremstyle{remark}
\newtheorem{rem}{Remark}
\newcommand{\sgn}{\mathop{\mathrm{sgn}}}
\providecommand{\abs}[1]{\left\vert#1\right\vert}
\providecommand{\res}[1]{\Res\displaylimits_{#1}} 
\providecommand{\norm}[1]{\left\lVert#1\right\rVert}
\providecommand{\mtx}[1]{\mathbf{#1}}
\providecommand{\mean}[1]{E\left[ #1 \right]}
\providecommand{\fourier}{\overset{\mathcal{F}}{ \rightleftharpoons}}
\providecommand{\system}{\overset{\mathcal{H}}{ \longleftrightarrow}}
\newcommand{\solution}{\noindent \textbf{Solution: }}
\newcommand{\cosec}{\,\text{cosec}\,}
\newcommand*{\permcomb}[4][0mu]{{{}^{#3}\mkern#1#2_{#4}}}
\newcommand*{\perm}[1][-3mu]{\permcomb[#1]{P}}
\newcommand*{\comb}[1][-1mu]{\permcomb[#1]{C}}
\renewcommand{\thetable}{\arabic{table}} 
\providecommand{\dec}[2]{\ensuremath{\overset{#1}{\underset{#2}{\gtrless}}}}
\newcommand{\myvec}[1]{\ensuremath{\begin{pmatrix}#1\end{pmatrix}}}
\newcommand{\mydet}[1]{\ensuremath{\begin{vmatrix}#1\end{vmatrix}}}
\numberwithin{equation}{section}
\numberwithin{figure}{section}
\numberwithin{table}{section}
\makeatletter
\@addtoreset{figure}{problem}
\makeatother
\let\StandardTheFigure\thefigure{}
\let\vec\mathbf{}
\def\putbox#1#2#3{\makebox[0in][l]{\makebox[#1][l]{}\raisebox{\baselineskip}[0in][0in]{\raisebox{#2}[0in][0in]{#3}}}}
     \def\rightbox#1{\makebox[0in][r]{#1}}
     \def\centbox#1{\makebox[0in]{#1}}
     \def\topbox#1{\raisebox{-\baselineskip}[0in][0in]{#1}}
     \def\midbox#1{\raisebox{-0.5\baselineskip}[0in][0in]{#1}}
\vspace{3cm}
\title{Assignment 5 12th Class}
\author{Gunjit Mittal (AI21BTECH11011)}
\maketitle
Download all python codes from 
\begin{lstlisting}
https://github.com/GunjitMittal/Assignment5/tree/main/Assignment5/code
\end{lstlisting}
Download all latex codes from 
\begin{lstlisting}
https://github.com/GunjitMittal/Assignment5/tree/main/Assignment5
\end{lstlisting} 
\section{Question}
Let A and B be independent events with $\Pr(A) = 0.3$ and $\Pr(B) = 0.4$. Find
\begin{enumerate}[label= (\roman{enumi})]
    \item $\Pr(A \cap  B)$ 
    \item $\Pr(A \cup  B)$
    \item $\Pr(A|B)$
    \item $\Pr(B|A)$
\end{enumerate}
\section{Solution}
\solution{}
\begin{align}
    \Pr(A+B) = \Pr(A) + \Pr(B) - \Pr(AB)
    \label{eq:union}
\end{align}
\begin{align}
    \Pr(A|B) = \frac{\Pr(AB)}{\Pr(B)}
    \label{eq:conditional}
\end{align}
\begin{enumerate}[label= (\roman{enumi})]
    \item $\Pr(AB) = \Pr(A) \times \Pr(B)$ cause A and B are independent events\\
    \begin{align}
        \Pr(AB) = 0.3 \times 0.4 = 0.12
    \end{align}
    \item Using\eqref{eq:union}\\
    \begin{align}
        \Pr(A+B) = 0.3 + 0.4 - 0.12 = 0.58
    \end{align}
    \item Using\eqref{eq:conditional}\\
    \begin{align}
        \Pr(A|B) = \frac{\Pr(AB)}{\Pr(B)} = \frac{0.12}{0.4} = 0.3
    \end{align}
    \item Using\eqref{eq:conditional}\\
    \begin{align}
        \Pr(B|A) = \frac{\Pr(AB)}{\Pr(A)} = \frac{0.12}{0.3} = 0.4
    \end{align}
\end{enumerate}
\noindent \textbf{Note:} Derivation of \eqref{eq:union} using Boolean Algebra:
\noindent We note that for any events A and B we have the following:
\begin{multline}
A + B = A(B + B') + B(A + A') \\ 
= (AB + BA) + AB' + A'B \\
= AB + AB' + A'B
\label{eq:id1}
\end{multline}
and
\begin{align}
\pr{A} &= \pr{AB'} + \pr{AB} 
\label{eq:id2}
\end{align}
Thus, taking probabilities in \eqref{eq:id1}, since all the events are mutually disjoint,
\begin{multline}
\pr{A + B} = \pr{AB'} + \pr{AB} + \pr{A'B} \\
= \pr{A} + \pr{A'B} \\
= \pr{A} + \pr{B} - \pr{AB}
\label{eq:derive}
\end{multline}
\end{document}  